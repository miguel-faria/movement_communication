\documentclass[a4paper,11pt,times,doublespace]{article}

\usepackage{amsmath}
\usepackage{mathtools}
\usepackage{calrsfs}

\title{Multi-User Legibility - The Math Behind It}
\author{Miguel Faria \& Francisco S. Melo}
	
\begin{document}
	
\maketitle
		
\section{Multi-User Legibility}

Multi-User legibility aims at improving the expressibility of a robot in a multi-user scenario by improving the legibility metric for the group the robot is working with, instead of each user independently.

The process to optimize legibility for multiple users combines the legibility in the perspective of each user into a single overall legibility metric and uses that to compute the gradient for the improvement of the robot's movement.

Thus, the optimization process first projects the robot's trajectory to each user's point of view; then computes the trajectory cost, legibility and legibility gradient for the trajectory for said user and finally combines all values into a single gradient update value for the trajectory.

\subsection{Legibility}

The legibility metric measures how much a robot's movement is understandable by a human without the human knowing the robot's objective beforehand. This metric encompasses how a user assigns the probability of the robot moving towards a target, giving more weight to the earlier portions of the trajectory, and is defined as
%
\begin{equation*}
	Legibility(\xi) = \frac{\int P(G_R|\xi_{S\to\xi(t)})f(t)dt}{\int f(t) dt},
\end{equation*}
%
with
%
\begin{equation*}
	P(G_R|\xi_{S\to\xi(t)}) = \frac{1}{\rho} \int \frac{\exp(-C(\xi_{S\to\xi(t)}) - V_{G_R}(\xi(t)))}{\exp(-V_{G_R}(S))} P(G_R),
\end{equation*}
%
where $\rho$ is a normalization constant, $C(x)$ is the cost function that gives the trajectory cost and $\xi_{S\to\xi(t)}$ is the trajectory from the start until time $t$. $V_{G_R}(q))$ is the $\min_{\xi \in \Xi_{q \rightarrow G_R}} C(\xi)$.

\subsubsection{Cost Function}

The cost function, $C(\xi)$, is the sum of squared velocities across the trajectory $\xi$, i.e.,

\begin{equation*}
	C(\xi) = \frac{1}{2} \sum_{t=1}^{n+1} \left\| \frac{q_{t+1} - q_t}{\triangle t} \right\|^2,
\end{equation*}
%
which takes the general form
%
\begin{equation*}
C(\xi)=\frac{1}{2}\xi^TA\xi + \xi^Tb + c,
\end{equation*}
%
where
%
\begin{align*}
	A&= K^TK, & b&= K^Te.
\end{align*}
%
In the expression above, $K$ is a $(n_{points}-2)\times(n_{points}-1)$ matrix and $e$ is a $(n_{points}-2)\times 1$ vector, respectively given by
%
\begin{align*}
	K&= \begin{bmatrix}
		 1 &  0 & 0 & \ldots &  0 &  0 \\
		-1 &  1 & 0 & \ldots &  0 &  0 \\
		 0 & -1 & 1 & \ldots &  0 &  0 \\
		\vdots  \\
		 0 &  0 & 0 & \ldots & -1 &  1 \\
		 0 &  0 & 0 & \ldots &  0 & -1 \\
	\end{bmatrix} &
	e&= \begin{bmatrix}
		q_{start} \\
		0 \\
		\vdots\\
		0\\
		q_{goal}
	\end{bmatrix}
\end{align*}

\subsection{Viewpoint Perspective}

In order to obtain the trajectory cost and legibility metrics for each user's point of view, first we have to project the robot's trajectory into the user's viewpoint. This is obtained by chaining a coordinate system transformation and a perspective projection from 3D to 2D.

\subsubsection{Coordinate Transformation}

To transform between coordinate systems, we perform a simple rotation and translation transformation between the world space, where the trajectory is described in, to each user's space.

Thus we apply the transformation $P^{U_n} = {}_W^{U_n}T \cdot P^W$ to all points in the trajectory. Where ${}_W^{U_n}T$ is a linear operator with the following structure:
	
	\begin{equation*}
		{}_W^{U_n}T = \begin{bmatrix}
			R_{x1} & R_{y1} & R_{z1} & T_x\\	
			R_{x2} & R_{y2} & R_{y2} & T_x\\
			R_{x3} & R_{y3} & R_{z3} & T_x\\
			0 & 0 & 0 & 1
		\end{bmatrix}
	\end{equation*}
	
This way, each point in the user's coordinate system is calculated:
	
	\begin{equation*}
		P^{U_n} = \begin{bmatrix} x_{U_n} \\ y_{U_n} \\ z_{U_n} \\ 1 \end{bmatrix} = 
		\begin{bmatrix}
		R_{x1_{U_n}} & R_{y1_{U_n}} & R_{z1_{U_n}} & T_{x_{U_n}}\\	
		R_{x2_{U_n}} & R_{y2_{U_n}} & R_{z2_{U_n}} & T_{y_{U_n}}\\
		R_{x3_{U_n}} & R_{y3_{U_n}} & R_{z3_{U_n}} & T_{z_{U_n}}\\
		0 & 0 & 0 & 1
		\end{bmatrix} \cdot
		\begin{bmatrix} x_W \\ y_W \\ z_W \\ 1 \end{bmatrix} ==>
	\end{equation*}
	\begin{equation*}
		P^{U_n} = \begin{bmatrix} x_{U_n} \\ y_{U_n} \\ z_{U_n} \\ 1 \end{bmatrix} = 
		\begin{bmatrix}
			R_{x1_{U_n}} \times x_W + R_{y1_{U_n}} \times y_W + R_{z1_{U_n}} \times z_W + T_{x_{U_n}}\\
			R_{x2_{U_n}} \times x_W + R_{y2_{U_n}} \times y_W + R_{z2_{U_n}} \times z_W + T_{y_{U_n}}\\
			R_{x3_{U_n}} \times x_W + R_{y3_{U_n}} \times y_W + R_{z3_{U_n}} \times z_W + T_{z_{U_n}}\\
			1
		\end{bmatrix}
	\end{equation*}

\subsubsection{2D Projection (Viewpoint Transformation)}

Having all points transformed into a user's coordinate system is easy to project them into his viewpoint by first applying the transformation $P^{2D} = {}^{2D}_{3D}T \cdot P^{3D}$, with ${}^{2D}_{3D}T$:
	\begin{equation*}
			{}^{2D}_{3D}T = \begin{bmatrix}
				S & 0 & 0 & 0\\	
				0 & S & 0 & 0\\
				0 & 0 & N & 1\\
				0 & 0 & F & 0
			\end{bmatrix}
	\end{equation*}

With:
	\begin{alignat*}{3}
		S = \frac{1}{\tan(\frac{fov}{2}\times\frac{\pi}{180})} &\quad
		N = -\frac{far\_plane}{far\_plane - near\_plane} &\quad
		F = \frac{far\_plane \times near\_plane}{far\_plane - near\_plane}
	\end{alignat*}

This results in:
	\begin{equation*}
		P^{2D} = 
		\begin{bmatrix} x_{2D} \\ y_{2D} \\ z_{2D} \\ w_{2D} \end{bmatrix} = 
		\begin{bmatrix}
			S & 0 & 0 & 0\\	
			0 & S & 0 & 0\\
			0 & 0 & N & -1\\
			0 & 0 & F & 0
		\end{bmatrix} \cdot
		\begin{bmatrix} x_{3D} \\ y_{3D} \\ z_{3D} \\ 1 \end{bmatrix} ==>
		\begin{bmatrix} x_{2D} \\ y_{2D} \\ z_{2D} \\ w_{2D} \end{bmatrix} = 
		\begin{bmatrix}
			S \times x_{3D}\\
			S \times y_{3D}\\
			N \times z_{3D} - 1\\
			F \times z_{3D}
		\end{bmatrix}
	\end{equation*}

With the $P^{2D}$ projection, we then take the $w$ coordinate and divide $x$ and $y$ by $w$ to maintain homogeneity in the dimensions and then by the $z$ coordinate to remap said coordinate.

	\begin{alignat*}{2}
		x_{2D} = \frac{\frac{x_{2D}}{w_{2D}}}{z_{2D}} &\quad
		y_{2D} = \frac{\frac{y_{2D}}{w_{2D}}}{z_{2D}} &\quad
	\end{alignat*}

\subsubsection{Generating Legibility - Trajectory Optimization}	

In order to generate a legible trajectory, we use a gradient ascent approach starting with a straight line trajectory. Using a usual gradient ascent approach:
\begin{equation*}
	\xi_{i+1} = \xi_i + \frac{1}{\eta} M^{-1} \nabla \mathcal{L}_V(\xi(t))
\end{equation*}

Where $\nabla \mathcal{L}_V(\xi(t))$ is the gradient of the legibility in the viewport for each user. $\nabla \mathcal{L}_V(\xi(t)) = \nabla \mathcal{L}(\xi(t)) \cdot \nabla_{3D}^{2D} T(\xi(t))$

$\nabla \mathcal{L}(\xi(t))$ is the gradient of the legibility metric for the trajectory.

\begin{equation*}
		\nabla \mathcal{L}(\xi(t)) = \nabla Legibility(\xi) = \frac{1}{\int f(t) dt}\frac{g'h - h'g}{h^2} P(G_R) f(t) \\
\end{equation*}
with:
\begin{equation*}
	g = \exp(V_{G_R}(S) - V_{G_R}(Q))
\end{equation*} 

\begin{equation*}
	h = \sum_G \exp(V_G(S) - V_G(Q))P(G)
\end{equation*} 
thus:
\begin{multline*}
	g'h - h'g = - V_{G_R}'(\xi)\,\exp(V_{G_R}(S) - V_{G_R}(\xi)) \sum_G \exp(V_G(S) - V_G(\xi))P(G)\\
	 - \exp(V_{G_R}(S) - V_{G_R}(\xi)) \sum_G - V_G'(\xi) \, \exp(V_G(S) - V_G(\xi))P(G)\\
	 = \exp(V_{G_R}(S) - V_{G_R}(\xi)) \sum_G \Big[ (-\exp(V_G(S)-V_G(\xi)) P(G) V_{G_R}'(\xi)) - \\(-V_G'(\xi)\exp(V_G(S)-V_G(\xi))P(G))\Big]\\
	 = \exp(V_{G_R}(S) - V_{G_R}(\xi)) \sum_G \frac{\exp(-V_G(\xi))P(G)}{\exp(-V_G(S))}(V_G'(\xi) - V_{G_R}'(\xi)) \qquad \qquad
\end{multline*}
combining:
\begin{multline*}
	\frac{g'h - h'g}{h^2} = \frac{\exp(V_{G_R}(S) - V_{G_R}(\xi)) \sum_G \frac{\exp(-V_G(\xi))P(G)}{\exp(-V_G(S))}(V_G'(\xi) - V_{G_R}'(\xi))}{(\sum_G \exp(V_G(S) - V_G(Q))P(G))^2}
\end{multline*}
resulting in:
\begin{multline*}
	\nabla \mathcal{L}(\xi(t)) = \frac{1}{\int f(t) dt} \frac{\exp(V_{G_R}(S) - V_{G_R}(\xi)) \sum_G \frac{\exp(-V_G(\xi))P(G)}{\exp(-V_G(S))}(V_G'(\xi) - V_{G_R}'(\xi))}{(\sum_G \exp(V_G(S) - V_G(Q))P(G))^2} P(G_R) f(t) \\
	= \frac{\exp(V_{G_R}(S) - V_{G_R}(\xi)) \sum_G \frac{\exp(-V_G(\xi))P(G)}{\exp(-V_G(S))}(V_G'(\xi) - V_{G_R}'(\xi))}{\int f(t)dt (\sum_G \exp(V_G(S) - V_G(Q))P(G))^2} P(G_R) f(t) 
\end{multline*}


$\nabla_{3D}^{2D} T(\xi(t))$ is the gradient of the projection from 3D to 2D, applied to every point in the trajectory.

\begin{equation*}
	\nabla_{3D}^{2D} T(\xi(t)) = \begin{bmatrix}
		\frac{S}{(F N z_t^2 - F z_t)} & 0 & \frac{S x_t (2 N z_t - 1)}{N^2 F z_t^4 - 2 N F z_t^3 - F z_t^2}\\\\
		0 & \frac{S}{(F N z_t^2 - F z_t)} & \frac{S y_t (2 N z_t - 1)}{N^2 F z_t^4 - 2 N F z_t^3 - F z_t^2}
	\end{bmatrix}
\end{equation*}

\end{document}