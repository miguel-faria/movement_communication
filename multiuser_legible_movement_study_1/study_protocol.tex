\documentclass[a4paper,11pt,times,doublespace]{article}

\usepackage{amsmath}
\usepackage{mathtools}
\usepackage{calrsfs}

\title{Multi-User Legibility - Study Protocol}
\author{Miguel Faria}
	
\begin{document}
	
\maketitle

\section{Description}

The aim of this study is to analyze the benefits of optimizing a robot's movement, during a multi-user collaboration, considering all the users involved over optimizing for each one individually.\\\\
%
To perform the study, we will use the M-Turk platform from Amazon, where each participant of the study will see different videos of a robot moving to grasp a cup from the same point of view.\\\\
%
In each video, the participant will be asked to predict which cup the robot is going to grasp. In order to do that, the video will be stopped in three different time points. At that time, and with the movement seen until then, the participant will have to guess which cup the robot is going to grasp.\\\\
%
To compare if it is better to optimize movements taking into consideration all users or each user individually, each participant will watch two four videos: one video with the movement being optimized for all users and three others, each of them, optimized for each participant's point of view.

\section{Study Procedure}

The study will take the form of a questionnaire administered to each participant on Amazon's M-Turk.\\\\
%
The questionnaire starts with a description of the study's objective on the first page. This description tells the participants they will see four videos, each showing a different movement that is going to one of the three cups they see in the image, with repetitions possible. The description also explains to the participants that the questionnaire they will fill out is aimed at evaluating different ways of a robot to convey intention using only movement and thus each video has the movement optimized according to one of those ways.\\\\
%
After the description, the participant will find some demographic questions. These questions will collect data regarding the participants' age, occupation, education and location.\\\\
%
After the demographic questions comes the questionnaire itself. At the start of each page there will be an informative note warning the participant that for each video, the participant will watch three segments of the movement and at the end select which cup the robot is going to grab.\\\\
%
After watching all the videos, they will answer four final questions, in which they will compare and choose what movement would be easier to work with and to understand the robot's intention.

\end{document}